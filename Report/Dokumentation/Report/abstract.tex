Until now drones keeps getting bigger and larger to carry bigger batteries with more capacity and to lift heavier payloads. This leads to drones getting less efficient, less responsive and gets more dangerous. Instead it has become popular to make drones smaller and increase the number of drones needed to solve a task. \\

Materials \& methods \\
This thesis describes how to make three drones follow a leader drone with a preprogrammed path as an example of drones cooperating. A Linux PC running MarkerLocator tracks each drones position and wirelessly transmits, using Xbee, the drones positions to all drones.The position of each drone is spoofed into the drone using the CAN-bus and thereby overwriting the onboard GPS. An outdoor test has been made using the onboard GPS to test the leader-follower algorithm in a bigger scale. A small PCB has been developed and mounted on each drone to route packages from the Xbee module to the CAN-bus of the drone and to measure the local altitude of the drone using a ultrasonic sensor. The PCB carries an AT90CAN128 as microcontroller which build-in CAN support making it obsolete to carry an external USB CAN-controller.\\


Results -> discussion\\
The accuracy of the vision based localisation is measured using a laser pointer pointing out the drones 2D position on the floor making it possible to measure the variance of the drones position. The leader-follower algorithm was also tested outside using the onboard GPS. The performance of the leader-follower algorithm is measured and discussed using plots that reveals the distance between the drones.\\

Conclusion -> perspective\\
It is shown that it is possible to implement the leader-follower algorithm using a vision and ultrasonic based positioning system. The distance between all drones when flying indoor was +/- 10 cm which is less than the maximum accepted error. It was possible to add 5 follower-drones without editing the code showing it is a generic system The system can further be used to indoor testing of navigation algorithms and explore the many possibilities drones has to offer. If drones at some point needs to fly indoor to help eg. Mobile robots navigating, vision might be a way to obtain a absolute indoor position for the drones.