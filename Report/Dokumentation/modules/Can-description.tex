\begin{table}[H]
\resizebox{\textwidth}{!}{%
	\begin{tabular}{|c|c|c|c|c|c|c|c|}
	\hline
	\multicolumn{8}{|c|}{CAN Message} \\ 
	\hline
	 \multicolumn{7}{|c|}{Identfier bits [29 bits]} &  		\multicolumn{1}{c|}{Data bits [max 64 bits]} \\
	 \hline
	 LCC [28:27] & TT [26] & FID [25:22] & DOC [21:16] & SOID 	[15:11] & TID [10:16] & SEID [5:0] & \\
	\hline
\end{tabular}}
	\caption{Table shows the identifier bits used in AutoQuad CAN messages}
	\label{tab:can_identifier_bits}
\end{table}

In figure \ref{tab:abbri_can_msg} the abbreviations can be seen.
\begin{table}[H]
		\begin{tabular}{|l|l|}
		\hline
		LCC & Logical Communications Channe \\
\hline
		TT & Target Type \\
\hline
		FID & Function ID \\
\hline
		DOC & Data Object Code \\
\hline
		SOID & Source ID \\
\hline
		TID & Target ID \\
\hline
		SEID & Sequence ID \\
\hline
		\end{tabular}
		\caption{Table shows 
abbreviations used in table \ref{tab:can_identifier_bits}}
		\label{tab:abbri_can_msg}
\end{table}

Each of the elements in an AQ messages will be explained how they are used in AQ. \\